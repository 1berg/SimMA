 \documentclass[paper=a4, pagesize, DIV=calc, BCOR=12.5mm, twoside=off, onecolumn=on, open = any, titlepage =on, parskip =half-, headsepline = on, footsepline = on, chapterprefix = on, appendixprefix = off, fontsize = 12pt, numbers = noenddot, abstract = on]{scrbook}
\usepackage[utf8]{inputenc}
\usepackage[german]{babel}
\usepackage{amsmath}
\usepackage{amsthm}
\usepackage{amsfonts}
\usepackage{amssymb}
\usepackage{makeidx}
\usepackage{graphicx}
\usepackage{tikz}
\usepackage{wrapfig}
\usepackage{geometry}
\usepackage{parallel}
\usepackage{todonotes}
 \usepackage{mathptmx}
 \usepackage{pdfsync}
 \usepackage{url}
 \usepackage{float}
 \usepackage{hyperref}



 %\usepackage[scaled=.90]{helvet}
 \usepackage[T1]{fontenc} 
%\newcommand{\changefont}[3]{ \fontfamily{#1} \fontseries{#2} \fontshape{#3} \selectfont}
%\changefont{cmr}{m}{n} %ppl for Palatino, ptm for Times New Roman
\setkomafont{title}{\rmfamily \bfseries}
\setkomafont{chapterentry}{\rmfamily \bfseries}
\usepackage{cutwin}
\usepackage[raggedleft]{titlesec}
\titlelabel{\thetitle.\quad}
\titleformat*{\chapter}{\scshape\bfseries\large}
\titleformat*{\section}{\bfseries\large}
\titleformat*{\subsection}{\bfseries \normalsize}
\titleformat*{\subsubsection}{\bfseries \normalsize}








\numberwithin{equation}{chapter}
%\usepackage{fancyhdr}
%\pagestyle{fancy}
%\fancyhead{}
%\fancyhead[RO,LE]{\thepage}
%\fancyhead[RE]{\thechapter }
%\fancyhead[LO]{\thesection }
%\fancyfoot{}
%\fancyfoot[LE,RO]{Pamina M. Berg}  %Hier muss noch etwas geändert werden...
\usepackage{blindtext}
\usepackage{todonotes}

%\usepackage[german]{babel}
\usepackage[german]{babel}
\usepackage{setspace}


\begin{document}
\thispagestyle{plain}

\pagenumbering{Roman}

\title{Entwicklung einer Java Simulationsumgebung für Lego Mindstorms NXT Roboter}
\subtitle{Exposé - Version 2}
\author{\emph{Pamina Maria Berg}\\
\emph{Matr.Nr. 6087438}}

\maketitle

\pagenumbering{arabic}
\par \singlespacing
\section*{Motivation}
\onehalfspacing
Die Arbeit mit Robotern bereichert den Informatikunterricht und das Nachmittagsprogramm vieler Schulen seit Jahren. Auf spielerische Art und Weise sollen Schülerinnen und Schüler (im Folgenden SuS abgekürzt) mit einfachen Konstrukten der objektorientierten Programmierung umzugehen lernen. Dies geschieht zum einen mit Drag-and-Drop Softwareangeboten, wie das standardmäßig mit ausgelieferte NXT Mindstorms Tool oder auch Enchanting. Zum anderen bietet sich ab der Mittelstufe (Klasse 7 -- 10) die Arbeit mit BlueJ zur Erstellung erster selbstgeschriebener Programme an. Hierzu kann eine BlueJ Extension genutzt werden, die mit der Java Virtual Machine leJOS NXJ für NXT Robotern arbeitet.\\
Doch immer wieder stoßen Lehrkräfte in den Schulen auf Hardwareprobleme jeglicher Art, wie zum Beispiel das Fehlen von einer ausreichenden Anzahl an Robotern im Unterricht, fehlende Firmwareupdates oder defekte Sensoren.\\
Auch ist das Zusammen- und wieder Auseinanderbauen der Roboter ein Aufwand, der nicht für den alltäglichen Unterricht geeignet sind. Um kleine Aufgaben mit Java zu lösen, wie zum Beispiel das Fahren einer S-Kurve oder das Anhalten auf einem bestimmten Punkt, muss bisher immer erst der Roboter gestartet, der Code in BlueJ verfasst, auf den Roboter übertragen und dieser dann zu einem geeigenete Parcours gebracht werden.\\
Um diese Probleme zu umgehen und den Einstieg in die objektorientierte Programmierung über die Benutzung von LEGO Mindstorm Robotern zu ermöglichen, soll im Rahmen dieser Masterarbeit eine Simulationsumgebung für die Arbeit mit LEGO Mindstorms NXT Roboter entstehen.

\par \singlespacing
\section*{Hintergrund / Entwicklung der NXT Roboter}
\onehalfspacing
Die ersten computergesteuerten LEGO Produkte wurden bereits 1986 veröffentlicht. In einer Zusammenarbeit von LEGO Education und dem Massachussettes Institute of Technology (MIT) wurde LEGO TC LOGO entwickelt. Dies war eine spezielle Abwandlung der Programmiersprache LOGO, mit der zusammengesetzte LEGO-Modelle gesteuert werden konnten \cite{rolling:14}.\\
Die Entwicklung eines programmierbaren LEGO-Steins begann 1988 und erreichte ihren Höhepunkt mit der Vorstellung des ersten MINDSTORMS Systems im Januar 1998, bei der der LEGO MINDSTORMS RCX Intelligent Brick, ein Microcomputer und somit das Kernstück des RCX-Systems, und das Robotics Invention System im Museum of Modern Art in London vorgestellt wurden.\\
Bereits zwei Monate nach Verkaufsstart wurde die FIRST LEGO League (FLL) gegründet -- eine Zusammenarbeit zwischen LEGO und FIRST (For Inspiration and Recognition of Science and Technology), die den Grundstein für die heute noch bestehende Wettbewerbsliga legte \cite{rolling:14}.\\
Die Vorstellung und der Verkaufsstart der Nachfolge-Roboter des RCX-Systems, den LEGO MINDSTORMS NXT Robotern, fand im August 2006 statt. Diese damals neu entwickelten Roboter sind auch, dank eines Updates in 2009, fast zehn Jahre später noch in den Schulen und Universitäten zu finden. Im April 2005 fand die erste FLL Weltmeisterschaft in Atlanta, Georgia, statt und bis heute bieten die Weltmeisterschaften einen Anlaufpunkt für Jugendliche auf der ganzen Welt, die ihr Können und ihre Roboter auf die Probe stellen wollen \cite{lego}.

\par \singlespacing
\section*{Fragestellung}
\onehalfspacing
In dieser Masterarbeit soll der Prototyp einer Simulationsumgebung entwickelt werden. Diese Simulationsumgebung soll auf Basis der von leJOS NXJ angebotenen API den SuS eine Möglichkeit zur schnellen Überprüfung ihrer Lösungen durch eine Simulation am Computer geben.\\
Des Weiteren soll im Laufe der Implementation festgestellt werden, inwiefern eine Integration der Simulationsumgebung in BlueJ mithilfe einer Extension sinnvoll und realisierbar ist.\\
Abschließend wird die Arbeit einen kleinen Anteil an didaktischen Inhalten umfassen. Dieser soll das Forschungsthema in einen schulischen Kontext einordnen und somit die Anforderungen, die an eine solche Simulationsumgebung gestellt werden, aus zwei Perspektiven darstellen.\\

Forschungsfragen:\\
\begin{itemize}
\item Welche Anforderungen werden an eine solche Simulationsumgebung gestellt - aus Schülerperspektive, aus Lehrerperspektive, aus Entwicklerperspektive?
\item Welche Integrationsmöglichkeit bietet BlueJ und wie kann diese sinnvoll genutzt werden?
\end{itemize}

\par \singlespacing
\section*{Lösungsidee}
\onehalfspacing
Die generelle Idee ist es, eine eigene API für den Simulator zu programmieren, die exakt die gleichen Methoden anbietet, wie die leJOS NXJ API. Somit wäre gewährleistet, dass die SuS eine einheitliche Syntax lernen und nicht zwischen Methoden wechseln müssen, um zum Beispiel die beiden Motoren zu starten oder die Sensoren anzusteuern.\\
Hierbei soll ein Roboter als Objekt in einer Bitmap, die aus einer Pixelgrafik erzeugt wird und einen Parcours darstellt, auf dem Bildschirm angezeigt werden und die programmierte Lösung der SuS auf diesem Parcours simulieren.\\


\par \singlespacing
\section*{Geplante Aktivitäten}
\onehalfspacing
Zunächst verschaffe ich mir einen Überblick darüber, welche Klassen implementiert werden müssen, um ein Mindestmaß an Funktionen der in Simulationsumgebung sicherzustellen. Hierzu gehören die Klassen der Motor- und Sensor-Ports, eine Klasse für die Motoren und jeweils eine Klasse für die Licht-, Ultraschall- und Tastsensoren.\\
Ich werde zunächst drei Szenarien als Pixelgrafiken erstellen, die dann von einer Bildeinleserklasse als Parcours generiert werden können. Um nun die Visualisierung des Roboters als Akteur in dem Problem-Szenario zu verwirklichen, werde ich diesen zunächst als einfaches Dreieck implementieren.\\
Ein Problem wird darin bestehen, dass der Roboter an sich bei der leJOS NXJ API nicht als eigenständiges Objekt gesehen wird, sondern die beiden Motoren A und B zur Steuerung des Roboters direkt angesprochen werden. Das Roboter-Objekt der Simulation muss also so mit den beiden Motoren verbunden werden, dass dieses ein visuelles Feedback geben kann. Sobald der rechte Motor des Roboters angesprochen wird, sollte die visuelle Repräsentation auf dem Parcours direkt darauf reagieren. Zudem müssen die Sensoren an das Roboter-Objekt "`angebaut"' werden, was eine weitere Schwierigkeit darstellt. 

\par \singlespacing
\section*{Zeitplan}
\onehalfspacing
Oktober 2015 
\begin{itemize}
\item Herstellung der Parcours-Pixelgrafiken
\item	Übersicht über die zu verwendenden Klassen verschaffen
\item	Einleitung der Ausarbeitung sowie ersten geschichtlicher Hintergrund
\end{itemize}

November 2015 – Dezember 2015
\begin{itemize}
\item	Implementation des Roboter-Objekts und der ersten Klassen
\item	Festhalten der Implementationsschritte mithilfe von Git
\item	Theoretischen Hintergrund sowie Anforderungen an die Implementation in der Ausarbeitung niederschreiben
\item	Anmeldung der Arbeit
\end{itemize}

Weihnachtsferien 2015 
\begin{itemize}
\item	Fertigstellung der Implementation
\begin{itemize}
\item Entscheidung über Einbettung der Oberfläche in BlueJ als Extension
\end{itemize}
\item	Testung der Simulationsoberfläche
\item	Implementationsschritte sowie Beschreibung der Software in die Ausarbeitung einpflegen
\end{itemize}

Anfang Januar 2016
\begin{itemize}
\item Zwischenstandspräsentation im Abschlussarbeitenseminar
\item Abgabe erster Entwurf der Ausarbeitung
\end{itemize}

Januar 2016 – Anfang Februar 2016
\begin{itemize}
\item Letzte Korrekturen in der Implementation
\item	Einbau des Feedbacks aus der Präsentation
\item	Erste Verbesserung der Ausarbeitung
\end{itemize}

Mitte Februar 2016
\begin{itemize}
\item Vorbesprechung zur Abgabe der Ausarbeitung
\item Einbau der Verbesserungsvorschläge
\end{itemize}

Ende Februar / Anfang März 2016
\begin{itemize}
\item Abgabe der Masterarbeit
\item Anmeldung zur mündlichen Abschlussprüfung
\begin{itemize}
\item Ggf. weitere Anträge stellen
\end{itemize}
\end{itemize}

März 2016
\begin{itemize}
\item Vorbereitung der mündlichen Abschlussprüfung
\end{itemize}

1.April 2016
\begin{itemize}
\item Mündliche Abschlussprüfung zur Masterarbeit / Kolloquium
\end{itemize}


\begin{thebibliography}{ABCD}

\bibitem[And89]{anderson:89}
Brain D.O. Anderson, John B. Moore. \emph{Optimal Control. Linear Quadratic Methods}, Prentice-Hill, Englewood Cliffs, 1989

\bibitem[Åst09]{astrom:2009}
Karl Johan Åström, Richard M. Murray. \emph{Feedback Systems. An Introduction for Scientists and Engineers}, Princeton University Press, Princeton, 2009

\bibitem[Ath66]{athans:1966}
Michael Athans, Peter L. Falb. \emph{Optimal Control. An Introduction to the Theory and Its Applications}, McGraw-Hill, New York, 1966

\bibitem[Bell63]{bellman:1963}
Richard Bellman, ed. \emph{Mathematical Optimization Techniques}, University of California Press, Berkeley and Los Angeles, 1963

\bibitem[Bel08]{bellon:2008}
James Bellon. "Riccati Equations in Optimal Control Theory." \emph{Mathematics Theses}, Paper 46, Georgia State University, 2008

\bibitem[Bry99]{bryson:1999}
A. E. Bryson. "Time-Varying Linear-Quadratic Control."  \emph{Journal of optimization theory and applications}, Vol.100, No. 2, Plenum Publishing Corporation, pp.515-525, March 1999

\bibitem[Bur12]{burk:2012}
Rainer E. Burkard, Uwe T. Zimmermann. \emph{Einführung in die Mathematische Optimierung}, Springer-Verlag Berlin, Heidelberg, 2012

\bibitem[Cle78]{clements:78}
David J. Clements, Brian D.O. Anderson. "Singular Optimal Control: The Linear-Quadratic Problem."  \emph{Lecture Notes in Control and Information Sciences}. Ed. A.V. Balakrishnan, M. Thoma, Springer-Verlag Berlin, Heidelberg, New York, 1978

\bibitem[Hon08]{hong:2008}
Seokmin Hong, Yonghwan Oh, Young-Hwan Chang, Bum-Jae You. "Walking pattern generation for Humanoid robots with LQR and feedforward control method." \emph{Industrial Electronics, 2008. IECON 2008. 34th Annual Conference of IEEE}, Orlando, Florida, pp. 1698 - 1703, November 2008

\bibitem[Li06]{li:2006}
Perry Y. Li. \emph{Linear Quadratic Optimal Control} in \emph{Lecture: ME8281: Advanced Control System Design}, Department of Mechanical Engineering, University of Minnesota, pp.101-120, March 2006

\bibitem[Lib12]{lib:2012}
Daniel Liberzon. \emph{Calculus of variations and optimal control theory: a concise introduction}, Princeton University Press, Princeton, 2012

\bibitem[Kaj07]{kajita:2007}
Shuji Kajita, Hirohisa Hirukawa, Kazuhito Yokoi, Kensuke Harada. \emph{Humanoide Roboter - Theorie und Technik des Kuünstlichen Menschen}. Ed. Shuji Kajita, Akademische Verlagsgesellschaft, Berlin, 2007

\bibitem[Kil10]{kil:2010}
Sinan Kilicaslan, Stephen P. Banks. "Existence of Solutions of Riccati Differential Equations for Linear Time Varying Systems". \emph{2010 American Control Conference}, Mariott Waterfront, Baltimore, MD, USA, June 2010

\bibitem[Kui14]{kui:2014}
Scott Kuindersma, Frank Permenter, Russ Tedrake. "An Efficiently Solvable Quadratic Program for Stabilizing Dynamic Locomotion." \emph{Proceedings of the International Conference on Robotics and Automation (ICRA)} Hong Kong, China, May 2014

\bibitem[Pap12]{pap:2012}
Markos Papageorgiou, Marian Leibold, Martin Buss. \emph{Optimierung. Statische, dynamische, stochastische Verfahren für die Anwendung}, Springer-Verlag Berlin, Heidelberg, 2012

\bibitem[Sar04]{sardain:2004}
Philippe Sardain, Guy Bessonnet. "Forces Acting on a Biped Robot. Conter of Pressure - Zero Moment Point." \emph{IEEE Trans. Syst. Man Cybern. A., Syst. Humans}, Vol. 34, No. 5, September 2004

\bibitem[Tou]{tou}
Marc Toussaint. \emph{Robotics. Control Theory}, FU Berlin

\bibitem[Tri03]{tri:2003}
Michael S. Triantafyllou, Franz S. Hover. \emph{Maneuvering and Control of Marine Vehicles}, Department of Ocean Engineering, Massachusetts Institute of Technology, Cambridge, November 2003

\newpage

\bibitem[Wil91]{willems:1991}
Jacques L. Willems, Frank M. Callier. "The Infinite Horizon and the Receding Horizon LQ-Problems with Partial Stabilization Constraints." \emph{The Riccati Equation}. Ed. Sergio Bittani, Alan J. Laub, Jan C. Willems, Springer-Verlag Berlin, Heidelberg, 1991

\bibitem[Wir06]{wirsching:2006}
Günther Wirsching. \emph{	
Gewöhnliche Differentialgleichungen: eine Einführung mit Beispielen, Aufgaben und Musterlösungen}, B. G. Teubner Verlag / GWV Fachverlage GmbH, Wiesbaden, 2006

\end{thebibliography}

\end{document}