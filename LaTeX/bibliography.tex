\begin{thebibliography}{ABCD}

\renewcommand{\refname}{\normalsize Literaturverzeichnis}

\bibitem[Abe01]{abend:01}
Michael Abend. "'Robotik und Sensorik. Darstellungsschwerpunkt: Selbständige Entwicklung "`unscharfer"' Algorithmen zur räumlichen Orientierung (unter Verwendung des LEGO-Mindstorms-Systems)", \emph{Schriftliche Prüfungsarbeit zur zweiten Staatsprüfung für das Amt des Studienrats}, Berlin, 2001

\bibitem[Her12]{hertzberg:12}
Joachim Hertzberg, Kai Lingemann, Andreas Nüchter. \emph{Mobile Roboter. Eine Einführung aus Sicht der Informatik}, Springer-Verlag Berlin Heidelberg, 2012

\bibitem[Hub07]{hubwieser:07}
Peter Hubwieser. \emph{Didaktik der Informatik}, 3. Auflage, Springer-Verlag Berlin Heidelberg, 2007

\bibitem[Lego]{lego}
o.V. URL: \url{http://www.lego.com/en-us/mindstorms/history}, Abgerufen am 06.11.2015, LEGO, 2015

\bibitem[leJOS]{lejos}
o.V. URL: \url{http://www.lejos.org/nxj.php}, Abgerufen am 14.12.2015, leJOS Java for Lego Mindstorms, 2015

\bibitem[Nie99]{nievergelt:99}
Jürg Nievergelt. \emph{"Roboter programmieren" - ein Kinderspiel. Bewegt sich auch etwas in der Allgemeinbildung?}, Informatik-Spektrum 22:5, S. 364--375, 1999

\bibitem[Rol14]{rolling:14}
Mark Rollins. \emph{Beginning LEGO MINDSTORMS EV3}, Apress, Berkeley, CA, 2014

\bibitem[Sch04]{schreiber:04}
Rafael Schreiber. "Der Einsatz von LEGO-Mindstorms im Informatikunterricht der 11. Klasse der Leonard-Bernstein-Oberschule. Sicherung und Transfer grundlegender algorithmischer Strukturen in NQC.", \emph{Schriftliche Prüfungsarbeit im Rahmen der zweiten Staatsprüfung für das Amt des Studienrats}, Berlin, 2004

\bibitem[Sto01]{stolt:01}
Matthias Stolt. "Roboter im Informatikunterricht", 2001

\bibitem[Wag05]{wagner:05}
Oliver Wagner. "LEGO Roboter im Informatikunterricht. Eine Untersuchung zum Einsatz des LEGO-Mindstorms-Systems zur Steigerung des Kooperationsvermögens im Informatikunterricht eines Grundkurses (12. Jahrgang, 2. Lernjahr) der Otto-Nagel-Oberschule (Gymnasium)", \emph{Schriftliche Prüfungsarbeit im Rahmen der zweiten Staatsprüfung für das Amt des Studienrats}, Berlin, 2005

\end{thebibliography}