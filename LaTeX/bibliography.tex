\begin{thebibliography}{ABCDEF}

\renewcommand{\refname}{\normalsize Literaturverzeichnis}

\bibitem[Abe01]{abend:01}
Michael Abend. "'Robotik und Sensorik. Darstellungsschwerpunkt: Selbständige Entwicklung "`unscharfer"' Algorithmen zur räumlichen Orientierung (unter Verwendung des LEGO-Mindstorms-Systems)", \emph{Schriftliche Prüfungsarbeit zur zweiten Staatsprüfung für das Amt des Studienrats}, Berlin, 2001

\bibitem[Abt15]{abts:15}
Dietmar Abts. \emph{Grundkurs JAVA. Von Grundlagen bis zu Datenbank- und Netzwerkanwendungen}, 8., überarbeitete und erweiterte Auflage, Springer Vieweg, Wiesbaden, 2015

\bibitem[Aeg16]{aegidius:16}
o.V. \url{http://www.aplu.ch/home/apluhomex.jsp?site=27}, Abgerufen am 02.02.2016, Aegidius Plüss -- NxtJLib, 2016

\bibitem[Bar03]{barnes:03}
David J. Barnes, Michael Kölling. \emph{Objektorientierte Programmierung mit Java. Eine praxisnahe Einführung mit BlueJ}, Übersetzt von Axel Schmolitzky, Pearson Studium, München, 2003

\pagebreak 

\bibitem[Ber10]{berns:10}
Karsten Berns, Daniel Schmidt. \emph{Programmierung mit LEGO MINDSTORMS NXT. Robotersysteme, Entwurfsmethodik, Algorithmen}, Springer Heidelberg Dordrecht London New York, 2010

\bibitem[Bow12]{bowes:12}
David Bowes. \url{http://homepages.herts.ac.uk/~comqdhb/lego/bluej.php}, Abgerufen am 07.02.2016, Herfortshire, 2012, Lejos NXJ extension for BlueJ

\bibitem[BricxCC]{bricxcc}
o.V. URL: \url{http://bricxcc.sourceforge.net/}, Abgerufen am 02.02.2016

\bibitem[Ehm09]{ehmann:09}
Matthias Ehmann et al. \emph{Duden Informatik - Sekundarstufe I / 9./10. Schuljahr - Objektorientierte Programmierung mit BlueJ}, Duden Schulbuchverlag Berlin Mannheim, 2009

\bibitem[Her12]{hertzberg:12}
Joachim Hertzberg, Kai Lingemann, Andreas Nüchter. \emph{Mobile Roboter. Eine Einführung aus Sicht der Informatik}, Springer-Verlag Berlin Heidelberg, 2012

\bibitem[HH09]{oberstufe:09}
Behörde für Schule und Berufsbildung Hamburg (Hrsg.). \emph{Informatik -- Bildungsplan  Gymnasiale Oberstufe}, Hamburg, 2009

\bibitem[HH11]{gymsek1:11}
Behörde für Schule und Berufsbildung Hamburg (Hrsg.). \emph{Informatik Wahlfplichtfach -- Bildungsplan Gymnasium Sekundarstufe I}, Hamburg, 2011

\bibitem[HH14]{stsmittel:14}
Behörde für Schule und Berufsbildung Hamburg (Hrsg.). \emph{Informatik Wahlpflichtfach -- Bildungsplan Stadtteilschule Jahrgangsstufen 7 -- 11}, Hamburg, 2014

\bibitem[Hub07]{hubwieser:07}
Peter Hubwieser. \emph{Didaktik der Informatik}, 3. Auflage, Springer-Verlag Berlin Heidelberg, 2007


\bibitem[Lego]{lego}
o.V. URL: \url{http://www.lego.com/en-us/mindstorms/history}, Abgerufen am 06.11.2015, LEGO, 2015

\bibitem[leJOS]{lejos}
o.V. URL: \url{http://www.lejos.org/nxj.php}, Abgerufen am 14.12.2015, leJOS Java for Lego Mindstorms, 2015

\bibitem[Lil08]{lilienthal:08}
Carloa Lilienthal. "Komplexität von Softwarearchitekturen -- Stile und Strategien --", \emph{Dissertation im Fachbereich Informatik der Universität Hamburg}, Hamburg, 2008

\bibitem[Mos12]{moser:12}
Christian Moser. \emph{User Experience Design: Mit erlebniszentrierter Softwareentwicklung zu Produkten, die begeistern}, Springer-Verlag Berlin Heidelberg, 2012

\bibitem[Nie99]{nievergelt:99}
Jürg Nievergelt. \emph{"Roboter programmieren" - ein Kinderspiel. Bewegt sich auch etwas in der Allgemeinbildung?}, Informatik-Spektrum 22:5, S. 364--375, 1999

\bibitem[nxcEditor]{nxceditor}
o.V. URL: \url{http://nxceditor.sourceforge.net/index_german.html}, Abgerufen am 02.02.2016, nxcEditor

\bibitem[PHBern]{phbern}
o.V. URL: \url{http://www.java-online.ch/lego/index.php?inhalt_links=home/nav_home.inc.php&inhalt_mitte=home/home.inc.php}, Abgerufen am 02.02.2016, Lego-Robotik mit Java 

\pagebreak

\bibitem[Roberta]{roberta}
o.V. URL: \url{http://roberta-home.de/de/aktuelles/simulator-und-editor-f\%C3\%BCr-nxc-linux-windows-mac-os-x}, Abgerufen am 02.02.2016, Roberta -- Lernen mit Robotern, Simulator und Editor für NXC (Linux, Windows, MacOS X)

\bibitem[Rol14]{rolling:14}
Mark Rollins. \emph{Beginning LEGO MINDSTORMS EV3}, Apress, Berkeley, CA, 2014

\bibitem[Sch04]{schreiber:04}
Rafael Schreiber. "Der Einsatz von LEGO-Mindstorms im Informatikunterricht der 11. Klasse der Leonard-Bernstein-Oberschule. Sicherung und Transfer grundlegender algorithmischer Strukturen in NQC.", \emph{Schriftliche Prüfungsarbeit im Rahmen der zweiten Staatsprüfung für das Amt des Studienrats}, Berlin, 2004

\bibitem[Schwa07]{schwarzer:07}
Christine Schwarzer, Petra Buchwald. "{Umlernen und Dazulernen}.", In: Michael Göhlich, Christoph Wulf, Jörg Zirfas (Hrsg.): \emph{Pädagogische Theorien des Lernens}, Beltz, Weinheim und Basel,  S. 213--221, 2007


\bibitem[RWTH]{rwth}
o.V. URL:{http://schuelerlabor.informatik.rwth-aachen.de/simulator}, Abgerufen am 02.02.2016, Simulator für LEGO Mindstorms NXT Roboter

\bibitem[Sto01]{stolt:01}
Matthias Stolt. "Roboter im Informatikunterricht", 2001

\bibitem[Ull12]{ullenboom:12}
Christian Ullenboom. \emph{Java ist auch eine Insel -- Das umfassende Handbuch}, 10. Auflage, Galileo Press, Bonn, 2012

\bibitem[Wag05]{wagner:05}
Oliver Wagner. "LEGO Roboter im Informatikunterricht. Eine Untersuchung zum Einsatz des LEGO-Mindstorms-Systems zur Steigerung des Kooperationsvermögens im Informatikunterricht eines Grundkurses (12. Jahrgang, 2. Lernjahr) der Otto-Nagel-Oberschule (Gymnasium)", \emph{Schriftliche Prüfungsarbeit im Rahmen der zweiten Staatsprüfung für das Amt des Studienrats}, Berlin, 2005

\bibitem[Zül90]{züllighoven:90}
Reinhard Budde, Heinz Züllighoven. \emph{Software-Werkzeuge in einer Programmierwerkstatt. Ansätze eines hermeneutisch fundierten Werkzeug- und Maschinenbegriffs}, Oldenbourg, München [u.a.], 1990
\end{thebibliography}