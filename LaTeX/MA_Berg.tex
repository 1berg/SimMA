 \documentclass[paper=a4, pagesize, DIV=calc, BCOR=12.5mm, twoside=on, onecolumn=on, open = any, titlepage =on, parskip =half-, headsepline = on, footsepline = on, chapterprefix = on, appendixprefix = off, fontsize = 12pt, numbers = noenddot, abstract = on]{scrbook}
\usepackage[utf8]{inputenc}
\usepackage[german]{babel}
\usepackage{amsmath}
\usepackage{amsthm}
\usepackage{amsfonts}
\usepackage{amssymb}
\usepackage{makeidx}
\usepackage{graphicx}
\usepackage{tikz}
\usepackage{wrapfig}
\usepackage{geometry}
\usepackage{parallel}
\usepackage{todonotes}
 \usepackage{mathptmx}
 \usepackage{pdfsync}
 \usepackage{url}
 \usepackage{float}
 \usepackage{hyperref}



 %\usepackage[scaled=.90]{helvet}
 \usepackage[T1]{fontenc} 
%\newcommand{\changefont}[3]{ \fontfamily{#1} \fontseries{#2} \fontshape{#3} \selectfont}
%\changefont{cmr}{m}{n} %ppl for Palatino, ptm for Times New Roman
\setkomafont{title}{\rmfamily \bfseries}
\setkomafont{chapterentry}{\rmfamily \bfseries}
\usepackage{cutwin}
\usepackage[raggedleft]{titlesec}
\titlelabel{\thetitle.\quad}
\titleformat*{\chapter}{\scshape\bfseries\large}
\titleformat*{\section}{\bfseries\large}
\titleformat*{\subsection}{\bfseries \normalsize}
\titleformat*{\subsubsection}{\bfseries \normalsize}








\numberwithin{equation}{chapter}
%\usepackage{fancyhdr}
%\pagestyle{fancy}
%\fancyhead{}
%\fancyhead[RO,LE]{\thepage}
%\fancyhead[RE]{\thechapter }
%\fancyhead[LO]{\thesection }
%\fancyfoot{}
%\fancyfoot[LE,RO]{Pamina M. Berg}  %Hier muss noch etwas geändert werden...
\usepackage{blindtext}
\usepackage{todonotes}

%\usepackage[german]{babel}
\usepackage[german]{babel}
\usepackage{setspace}
\theoremstyle{definition}
\newtheorem{definition}{Definition}
\theoremstyle{plain}
\newtheorem{beispiel}{Example}
\theoremstyle{plain}
\newtheorem{satz}{Theorem}
\theoremstyle{remark}
\newtheorem{bemerkung}{Remark}
\theoremstyle{plain}
\newtheorem{cor}{Corollary}
\theoremstyle{plain}
\newtheorem{prop}{Proposition}

\begin{document}
\newpage
\thispagestyle{plain}

\pagenumbering{Roman}
\newcommand*\diff{\mathop{}\!\mathrm{d}}

\begin{titlepage}


\begin{figure}[htbp]
		\begin{minipage}[b]{25mm}
			\includegraphics[width=25mm,clip]{images/logo_uhh}
		\end{minipage}
		\begin{minipage}[b]{2mm}
			\includegraphics[width=1mm,height=25mm]{images/greypixel}
		\end{minipage}
		\begin{minipage}[b]{12.5cm}
			{   
				\vspace{2mm}
				{\Large Universität Hamburg } \\
				Fakultät für Mathematik,\\
				Informatik und Naturwissenschaften \\
				Department Informatik \\
			}
		\end{minipage}
	\end{figure}


\begin{center} 
\vspace{0.5cm} 
\begin{tabular}{c}
% \vspace{0.5cm}\\
\Large \textsc{Entwicklung einer Java Simulationsumgebung}\\
 \\
\Large \textsc{für Lego Mindstorms NXT Roboter}\\
\\
\\
\small Masterarbeit im Fach Informatik zur Erlangung des akademischen Grades\\
\\
\normalsize \textbf{Master of Education}\\
\\
\small im Studiengang Lehramt an Gymnasien M.Ed.\\
\\
\\
\large \textbf{Pamina Maria Berg}\\
\small \emph{Matr.Nr. 6087438}\\
\\
\normalsize Hamburg, den \today
\end{tabular}
\par 
\end{center}
\par
\vspace*{3cm}
\begin{tabular}{l}
\emph{Erstgutachter}\\
Dr.\, Guido Gryczan\\
\emph{Zweitgutachter}\\
Jun.-Prof.\, Dr. Maria Knobelsdorf\\
\emph{Betreuer}\\
Till Aust\\
Fredrik Winkler\\
\end{tabular}

\end{titlepage}


\thispagestyle{empty}
\cleardoublepage



\newpage
\listoffigures
\newpage
\tableofcontents
\thispagestyle{empty}
\cleardoublepage
\newpage
\pagenumbering{arabic}
\par \singlespacing
\chapter{Einleitung}
\onehalfspacing
Die Arbeit mit Robotern bereichert den Informatikunterricht und das Nachmittagsprogramm vieler Schulen seit Jahren. Auf spielerische Art und Weise sollen Schülerinnen und Schüler (im Folgenden SuS abgekürzt) mit einfachen Konstrukten der objektorientierten Programmierung umzugehen lernen. Dies geschieht zum einen mit Drag-and-Drop Softwareangeboten wie das standardmäßig mit ausgelieferte NXT Mindstorms Tool \todo{Referenz:NXT Software}, oder auch Enchanting. \todo{Referenz: Enchanting} Zum anderen bietet sich ab der Mittelstufe (Klasse 7 -- 10) die Arbeit mit BlueJ zur Erstellung erster selbstgeschriebener Programme an. Hierzu kann eine BlueJ Extension genutzt werden, die mit der Java Virtual Machine leJOS NXJ für NXT Robotern arbeitet.\\

Doch immer wieder stoßen Lehrkräfte in den Schulen auf Hardwareprobleme jeglicher Art, wie zum Beispiel das Fehlen von einer ausreichenden Anzahl an Robotern im Unterricht oder auch fehlende Firmwareupdates oder defekte Sensoren.\\
Auch ist das Zusammen- und wieder Auseinanderbauen der Roboter ein Aufwand, der nicht für den alltäglichen Unterricht geeignet sind. Um kleine Aufgaben mit Java zu lösen, wie zum Beispiel das Fahren einer S-Kurve oder das Anhalten einem bestimmten Punkt, muss bisher immer erst der Roboter gestartet, der Code in BlueJ verfasst, auf den Roboter übertragen und dieser dann zu einem geeigenete Parcours gebracht werden.\\

Um diese Probleme zu umgehen und den Einstieg in die objektorientierte Programmierung über die Benutzung von LEGO Mindstorm Robotern zu ermöglichen, soll im Rahmen dieser Masterarbeit eine Simulationsumgebung für die Arbeit mit LEGO Mindstorms NXT Roboter entstehen. \\todo{Referenz}
\newpage
\par\singlespacing
\chapter{Ausgangssituation}
%The following definitions and equations are adapted from \cite{li:2006}.
\onehalfspacing
\section{Das LEGO Mindstorms NXT System}

\section{Die Arbeit mit LEGO-Robotern im Unterricht}

\subsection{Verankerung im Bildungsplan}


\section{Bisher verfügbare Softwarelösungen}

\subsection{LEGO Mindstorms NXT}

\subsection{Enchanting}

\subsection{BlueJ}

\par \singlespacing
\section{leJOS}
\onehalfspacing



\par \singlespacing
\chapter{Anforderungen an die Neuimplementierung}
\onehalfspacing




\section{Schülerperspektive}
\onehalfspacing

\par \singlespacing
\section{Lehrkraft}
\label{sec:lehrkraft}
\par \onehalfspacing

\newpage
\par \singlespacing
\section{Erweiterbarkeit} \label{sec:erweiterbarkeit}
\onehalfspacing




\chapter{Entwickelte Software}



\section{Beschreibung der Software}



\section{Softwarearchitektur}

\newpage
\chapter{Zusammenfassung und Ausblick}

\newpage
\begin{thebibliography}{ABCD}

\bibitem[And89]{anderson:89}
Brain D.O. Anderson, John B. Moore. \emph{Optimal Control. Linear Quadratic Methods}, Prentice-Hill, Englewood Cliffs, 1989

\bibitem[Åst09]{astrom:2009}
Karl Johan Åström, Richard M. Murray. \emph{Feedback Systems. An Introduction for Scientists and Engineers}, Princeton University Press, Princeton, 2009

\bibitem[Ath66]{athans:1966}
Michael Athans, Peter L. Falb. \emph{Optimal Control. An Introduction to the Theory and Its Applications}, McGraw-Hill, New York, 1966

\bibitem[Bell63]{bellman:1963}
Richard Bellman, ed. \emph{Mathematical Optimization Techniques}, University of California Press, Berkeley and Los Angeles, 1963

\bibitem[Bel08]{bellon:2008}
James Bellon. "Riccati Equations in Optimal Control Theory." \emph{Mathematics Theses}, Paper 46, Georgia State University, 2008

\bibitem[Bry99]{bryson:1999}
A. E. Bryson. "Time-Varying Linear-Quadratic Control."  \emph{Journal of optimization theory and applications}, Vol.100, No. 2, Plenum Publishing Corporation, pp.515-525, March 1999

\bibitem[Bur12]{burk:2012}
Rainer E. Burkard, Uwe T. Zimmermann. \emph{Einführung in die Mathematische Optimierung}, Springer-Verlag Berlin, Heidelberg, 2012

\bibitem[Cle78]{clements:78}
David J. Clements, Brian D.O. Anderson. "Singular Optimal Control: The Linear-Quadratic Problem."  \emph{Lecture Notes in Control and Information Sciences}. Ed. A.V. Balakrishnan, M. Thoma, Springer-Verlag Berlin, Heidelberg, New York, 1978

\bibitem[Hon08]{hong:2008}
Seokmin Hong, Yonghwan Oh, Young-Hwan Chang, Bum-Jae You. "Walking pattern generation for Humanoid robots with LQR and feedforward control method." \emph{Industrial Electronics, 2008. IECON 2008. 34th Annual Conference of IEEE}, Orlando, Florida, pp. 1698 - 1703, November 2008

\bibitem[Li06]{li:2006}
Perry Y. Li. \emph{Linear Quadratic Optimal Control} in \emph{Lecture: ME8281: Advanced Control System Design}, Department of Mechanical Engineering, University of Minnesota, pp.101-120, March 2006

\bibitem[Lib12]{lib:2012}
Daniel Liberzon. \emph{Calculus of variations and optimal control theory: a concise introduction}, Princeton University Press, Princeton, 2012

\bibitem[Kaj07]{kajita:2007}
Shuji Kajita, Hirohisa Hirukawa, Kazuhito Yokoi, Kensuke Harada. \emph{Humanoide Roboter - Theorie und Technik des Kuünstlichen Menschen}. Ed. Shuji Kajita, Akademische Verlagsgesellschaft, Berlin, 2007

\bibitem[Kil10]{kil:2010}
Sinan Kilicaslan, Stephen P. Banks. "Existence of Solutions of Riccati Differential Equations for Linear Time Varying Systems". \emph{2010 American Control Conference}, Mariott Waterfront, Baltimore, MD, USA, June 2010

\bibitem[Kui14]{kui:2014}
Scott Kuindersma, Frank Permenter, Russ Tedrake. "An Efficiently Solvable Quadratic Program for Stabilizing Dynamic Locomotion." \emph{Proceedings of the International Conference on Robotics and Automation (ICRA)} Hong Kong, China, May 2014

\bibitem[Pap12]{pap:2012}
Markos Papageorgiou, Marian Leibold, Martin Buss. \emph{Optimierung. Statische, dynamische, stochastische Verfahren für die Anwendung}, Springer-Verlag Berlin, Heidelberg, 2012

\bibitem[Sar04]{sardain:2004}
Philippe Sardain, Guy Bessonnet. "Forces Acting on a Biped Robot. Conter of Pressure - Zero Moment Point." \emph{IEEE Trans. Syst. Man Cybern. A., Syst. Humans}, Vol. 34, No. 5, September 2004

\bibitem[Tou]{tou}
Marc Toussaint. \emph{Robotics. Control Theory}, FU Berlin

\bibitem[Tri03]{tri:2003}
Michael S. Triantafyllou, Franz S. Hover. \emph{Maneuvering and Control of Marine Vehicles}, Department of Ocean Engineering, Massachusetts Institute of Technology, Cambridge, November 2003

\newpage

\bibitem[Wil91]{willems:1991}
Jacques L. Willems, Frank M. Callier. "The Infinite Horizon and the Receding Horizon LQ-Problems with Partial Stabilization Constraints." \emph{The Riccati Equation}. Ed. Sergio Bittani, Alan J. Laub, Jan C. Willems, Springer-Verlag Berlin, Heidelberg, 1991

\bibitem[Wir06]{wirsching:2006}
Günther Wirsching. \emph{	
Gewöhnliche Differentialgleichungen: eine Einführung mit Beispielen, Aufgaben und Musterlösungen}, B. G. Teubner Verlag / GWV Fachverlage GmbH, Wiesbaden, 2006

\end{thebibliography}
\newpage
\thispagestyle{empty}
\vspace*{\fill}
"Hiermit versichere ich, dass ich die Arbeit selbstständig verfasst und keine anderen als die angegebenen Hilfsmittel – insbesondere keine im Quellenverzeichnis nicht benannten Internet-Quellen – benutzt habe, die Arbeit vorher nicht in einem anderen Prüfungsverfahren eingereicht habe und die eingereichte schriftliche Fassung der auf dem elektronischen Speichermedium entspricht."\\

Hamburg, \today \hspace*{\fill} \dots \dots \dots \dots \dots \dots \dots\\
\hspace*{\fill} Pamina Maria Berg $\,$
\end{document}